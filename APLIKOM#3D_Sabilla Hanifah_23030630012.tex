% Options for packages loaded elsewhere
\PassOptionsToPackage{unicode}{hyperref}
\PassOptionsToPackage{hyphens}{url}
\documentclass[
]{book}
\usepackage{xcolor}
\usepackage{amsmath,amssymb}
\setcounter{secnumdepth}{-\maxdimen} % remove section numbering
\usepackage{iftex}
\ifPDFTeX
  \usepackage[T1]{fontenc}
  \usepackage[utf8]{inputenc}
  \usepackage{textcomp} % provide euro and other symbols
\else % if luatex or xetex
  \usepackage{unicode-math} % this also loads fontspec
  \defaultfontfeatures{Scale=MatchLowercase}
  \defaultfontfeatures[\rmfamily]{Ligatures=TeX,Scale=1}
\fi
\usepackage{lmodern}
\ifPDFTeX\else
  % xetex/luatex font selection
\fi
% Use upquote if available, for straight quotes in verbatim environments
\IfFileExists{upquote.sty}{\usepackage{upquote}}{}
\IfFileExists{microtype.sty}{% use microtype if available
  \usepackage[]{microtype}
  \UseMicrotypeSet[protrusion]{basicmath} % disable protrusion for tt fonts
}{}
\makeatletter
\@ifundefined{KOMAClassName}{% if non-KOMA class
  \IfFileExists{parskip.sty}{%
    \usepackage{parskip}
  }{% else
    \setlength{\parindent}{0pt}
    \setlength{\parskip}{6pt plus 2pt minus 1pt}}
}{% if KOMA class
  \KOMAoptions{parskip=half}}
\makeatother
\usepackage{graphicx}
\makeatletter
\newsavebox\pandoc@box
\newcommand*\pandocbounded[1]{% scales image to fit in text height/width
  \sbox\pandoc@box{#1}%
  \Gscale@div\@tempa{\textheight}{\dimexpr\ht\pandoc@box+\dp\pandoc@box\relax}%
  \Gscale@div\@tempb{\linewidth}{\wd\pandoc@box}%
  \ifdim\@tempb\p@<\@tempa\p@\let\@tempa\@tempb\fi% select the smaller of both
  \ifdim\@tempa\p@<\p@\scalebox{\@tempa}{\usebox\pandoc@box}%
  \else\usebox{\pandoc@box}%
  \fi%
}
% Set default figure placement to htbp
\def\fps@figure{htbp}
\makeatother
\setlength{\emergencystretch}{3em} % prevent overfull lines
\providecommand{\tightlist}{%
  \setlength{\itemsep}{0pt}\setlength{\parskip}{0pt}}
\usepackage{bookmark}
\IfFileExists{xurl.sty}{\usepackage{xurl}}{} % add URL line breaks if available
\urlstyle{same}
\hypersetup{
  hidelinks,
  pdfcreator={LaTeX via pandoc}}

\author{}
\date{}

\begin{document}
\frontmatter

\mainmatter
\chapter{Tugas Aplikasi Komputer}\label{tugas-aplikasi-komputer}

Kelas : Matematika B 2023

NIM : 23030630012

\chapter{Menggambar Plot 3D dengan EMT}\label{menggambar-plot-3d-dengan-emt}

Ini adalah pengenalan terhadap plot 3D di Euler. Kita memerlukan plot 3D untuk memvisualisasikan fungsi dari dua variabel.

Euler menggambar fungsi tersebut menggunakan algoritma pengurutan untuk menyembunyikan bagian-bagian di latar belakang. Secara umum, Euler menggunakan proyeksi pusat. Standarnya adalah dari kuadran x-y positif ke arah titik asal x=y=z=0, tetapi sudut=0° terlihat dari arah sumbu y. Sudut pandang dan ketinggian dapat diubah.

Euler dapat memetakan

\begin{itemize}
\item
  permukaan dengan bayangan dan garis level atau rentang level,
\item
  awan titik-titik,
\item
  kurva parametrik,
\item
  permukaan implisit.
\end{itemize}

Plot 3D dari sebuah fungsi menggunakan plot3d. Cara termudah adalah memplot ekspresi dalam x dan y. Parameter r mengatur rentang plot di sekitar (0,0).

\textgreater aspect(1.5); plot3d(``x\^{}2+sin(y)'',-5,5,0,6*pi):

\begin{figure}
\centering
\pandocbounded{\includegraphics[keepaspectratio]{images/APLIKOM\#3D_Sabilla Hanifah_23030630012-001.png}}
\caption{images/APLIKOM\#3D\_Sabilla\%20Hanifah\_23030630012-001.png}
\end{figure}

\textgreater plot3d(``x\^{}2+x*sin(y)'',-5,5,0,6*pi):

\begin{figure}
\centering
\pandocbounded{\includegraphics[keepaspectratio]{images/APLIKOM\#3D_Sabilla Hanifah_23030630012-002.png}}
\caption{images/APLIKOM\#3D\_Sabilla\%20Hanifah\_23030630012-002.png}
\end{figure}

Silakan lakukan modifikasi agar gambar ``talang bergelombang'' tersebut tidak lurus melainkan melengkung/melingkar, baik melingkar secara mendatar maupun melingkar turun/naik (seperti papan peluncur pada kolam renang. Temukan rumusnya.

\chapter{Fungsi dari dua Variabel}\label{fungsi-dari-dua-variabel}

Untuk grafik sebuah fungsi, gunakan

\begin{itemize}
\item
  ekspresi sederhana dalam x dan y,
\item
  nama fungsi dari dua variabell
\item
  atau matriks data.
\end{itemize}

Standarnya adalah kisi-kisi kawat yang terisi dengan warna yang berbeda di kedua sisi. Perhatikan bahwa jumlah default interval grid adalah 10, namun plot menggunakan jumlah default 40x40 persegi panjang untuk membangun permukaan. Hal ini dapat diubah.

\begin{itemize}
\item
  n=40, n={[}40,40{]}: jumlah garis kisi di setiap arah
\item
  grid=10, grid={[}10,10{]}: jumlah garis grid di setiap arah.
\end{itemize}

Kami menggunakan default n=40 dan grid=10.

\textgreater plot3d(``x\textsuperscript{2+y}2''):

\begin{figure}
\centering
\pandocbounded{\includegraphics[keepaspectratio]{images/APLIKOM\#3D_Sabilla Hanifah_23030630012-003.png}}
\caption{images/APLIKOM\#3D\_Sabilla\%20Hanifah\_23030630012-003.png}
\end{figure}

Interaksi pengguna dapat dilakukan dengan parameter \textgreater user. Pengguna dapat menekan tombol berikut ini.

\begin{itemize}
\item
  kiri, kanan, atas, bawah: memutar sudut pandang
\item
  +,-: memperbesar atau memperkecil
\item
  a: menghasilkan anaglyph (lihat di bawah)
\item
  l: beralih memutar sumber cahaya (lihat di bawah)
\item
  spasi: mengatur ulang ke default
\item
  kembali: mengakhiri interaksi
\end{itemize}

\textgreater plot3d(``exp(-x\textsuperscript{2+y}2)'',\textgreater user, \ldots{}\\
\textgreater{} title=``Turn with the vector keys (press return to finish)''):

\begin{figure}
\centering
\pandocbounded{\includegraphics[keepaspectratio]{images/APLIKOM\#3D_Sabilla Hanifah_23030630012-004.png}}
\caption{images/APLIKOM\#3D\_Sabilla\%20Hanifah\_23030630012-004.png}
\end{figure}

Rentang plot untuk fungsi dapat ditentukan dengan

\begin{itemize}
\item
  a, b: rentang x
\item
  c, d: rentang y
\item
  r: bujur sangkar simetris di sekitar (0,0).
\item
  n: jumlah subinterval untuk plot.
\end{itemize}

Terdapat beberapa parameter untuk menskalakan fungsi atau mengubah tampilan grafik.

fscale: skala untuk nilai fungsi (standarnya adalah \textless fscale).

scale: angka atau vektor 1x2 untuk menskalakan ke arah x dan y.

frame: jenis bingkai (default 1).

\textgreater plot3d(``exp(-(x\textsuperscript{2+y}2)/5)'',r=10,n=80,fscale=4,scale=1.2,frame=3,\textgreater user):

\begin{figure}
\centering
\pandocbounded{\includegraphics[keepaspectratio]{images/APLIKOM\#3D_Sabilla Hanifah_23030630012-005.png}}
\caption{images/APLIKOM\#3D\_Sabilla\%20Hanifah\_23030630012-005.png}
\end{figure}

Tampilan dapat diubah dengan berbagai cara.

\begin{itemize}
\item
  jarak: jarak pandang ke plot.
\item
  zoom: nilai zoom.
\item
  angle: sudut ke sumbu y negatif dalam radian.
\item
  height: ketinggian tampilan dalam radian.
\end{itemize}

Nilai default dapat diperiksa atau diubah dengan fungsi view(). Fungsi ini mengembalikan parameter dalam urutan di atas.

\textgreater view

\begin{verbatim}
[5,  2.6,  2,  0.4]
\end{verbatim}

Jarak yang lebih dekat membutuhkan zoom yang lebih sedikit. Efeknya lebih seperti lensa sudut lebar.

Dalam contoh berikut ini, sudut = 0 dan tinggi = 0 terlihat dari sumbu y negatif. Label sumbu untuk y disembunyikan dalam kasus ini.

\textgreater plot3d(``x\^{}2+y'',distance=3,zoom=1,angle=pi/2,height=0):

\begin{figure}
\centering
\pandocbounded{\includegraphics[keepaspectratio]{images/APLIKOM\#3D_Sabilla Hanifah_23030630012-006.png}}
\caption{images/APLIKOM\#3D\_Sabilla\%20Hanifah\_23030630012-006.png}
\end{figure}

Plot terlihat selalu ke bagian tengah kubus plot. Anda dapat memindahkan bagian tengah dengan parameter center.

\textgreater plot3d(``x\textsuperscript{4+y}2'',a=0,b=1,c=-1,d=1,angle=-20°,height=20°, \ldots{}\\
\textgreater{} center={[}0.4,0,0{]},zoom=5):

\begin{figure}
\centering
\pandocbounded{\includegraphics[keepaspectratio]{images/APLIKOM\#3D_Sabilla Hanifah_23030630012-007.png}}
\caption{images/APLIKOM\#3D\_Sabilla\%20Hanifah\_23030630012-007.png}
\end{figure}

Plot diskalakan agar sesuai dengan kubus satuan untuk dilihat. Jadi, tidak perlu mengubah jarak atau melakukan zoom, tergantung pada ukuran plot. Namun demikian, label mengacu ke ukuran yang sesungguhnya.

Jika Anda menonaktifkannya dengan scale=false, Anda harus berhati-hati agar plot tetap muat di dalam jendela plotting, dengan mengubah jarak tampilan atau zoom, dan memindahkan bagian tengahnya.

\textgreater plot3d(``5*exp(-x\textsuperscript{2-y}2)'',r=2,\textless fscale,\textless scale,distance=13,height=50°, \ldots{}\\
\textgreater{} center={[}0,0,-2{]},frame=3):

\begin{figure}
\centering
\pandocbounded{\includegraphics[keepaspectratio]{images/APLIKOM\#3D_Sabilla Hanifah_23030630012-008.png}}
\caption{images/APLIKOM\#3D\_Sabilla\%20Hanifah\_23030630012-008.png}
\end{figure}

Plot polar juga tersedia. Parameter polar=true menggambar plot polar. Fungsi harus tetap merupakan fungsi dari x dan y. Parameter ``fscale'' menskalakan fungsi dengan skala sendiri. Jika tidak, fungsi akan diskalakan agar sesuai dengan kubus.

\textgreater plot3d(``1/(x\textsuperscript{2+y}2+1)'',r=5,\textgreater polar, \ldots{}\\
\textgreater{} fscale=2,\textgreater hue,n=100,zoom=4,\textgreater contour,color=blue):

\begin{figure}
\centering
\pandocbounded{\includegraphics[keepaspectratio]{images/APLIKOM\#3D_Sabilla Hanifah_23030630012-009.png}}
\caption{images/APLIKOM\#3D\_Sabilla\%20Hanifah\_23030630012-009.png}
\end{figure}

\textgreater function f(r) := exp(-r/2)*cos(r); \ldots{}\\
\textgreater{} plot3d(``f(x\textsuperscript{2+y}2)'',\textgreater polar,scale={[}1,1,0.4{]},r=pi,frame=3,zoom=4):

\begin{figure}
\centering
\pandocbounded{\includegraphics[keepaspectratio]{images/APLIKOM\#3D_Sabilla Hanifah_23030630012-010.png}}
\caption{images/APLIKOM\#3D\_Sabilla\%20Hanifah\_23030630012-010.png}
\end{figure}

Parameter rotate memutar fungsi dalam x di sekitar sumbu x.

\begin{itemize}
\item
  rotate = 1: Menggunakan sumbu x
\item
  rotate=2: Menggunakan sumbu z
\end{itemize}

\textgreater plot3d(``x\^{}2+1'',a=-1,b=1,rotate=true,grid=5):

\begin{figure}
\centering
\pandocbounded{\includegraphics[keepaspectratio]{images/APLIKOM\#3D_Sabilla Hanifah_23030630012-011.png}}
\caption{images/APLIKOM\#3D\_Sabilla\%20Hanifah\_23030630012-011.png}
\end{figure}

\textgreater plot3d(``x\^{}2+1'',a=-1,b=1,rotate=2,grid=5):

\begin{figure}
\centering
\pandocbounded{\includegraphics[keepaspectratio]{images/APLIKOM\#3D_Sabilla Hanifah_23030630012-012.png}}
\caption{images/APLIKOM\#3D\_Sabilla\%20Hanifah\_23030630012-012.png}
\end{figure}

\textgreater plot3d(``sqrt(25-x\^{}2)'',a=0,b=5,rotate=1):

\begin{figure}
\centering
\pandocbounded{\includegraphics[keepaspectratio]{images/APLIKOM\#3D_Sabilla Hanifah_23030630012-013.png}}
\caption{images/APLIKOM\#3D\_Sabilla\%20Hanifah\_23030630012-013.png}
\end{figure}

\textgreater plot3d(``x*sin(x)'',a=0,b=6pi,rotate=2):

\begin{figure}
\centering
\pandocbounded{\includegraphics[keepaspectratio]{images/APLIKOM\#3D_Sabilla Hanifah_23030630012-014.png}}
\caption{images/APLIKOM\#3D\_Sabilla\%20Hanifah\_23030630012-014.png}
\end{figure}

Berikut ini adalah plot dengan tiga fungsi.

\textgreater plot3d(``x'',``x\textsuperscript{2+y}2'',``y'',r=2,zoom=3.5,frame=3):

\begin{figure}
\centering
\pandocbounded{\includegraphics[keepaspectratio]{images/APLIKOM\#3D_Sabilla Hanifah_23030630012-015.png}}
\caption{images/APLIKOM\#3D\_Sabilla\%20Hanifah\_23030630012-015.png}
\end{figure}

\chapter{Plot Kontur}\label{plot-kontur}

Untuk plot, Euler menambahkan garis kisi-kisi. Sebagai gantinya, dimungkinkan untuk menggunakan garis level dan rona satu warna atau rona berwarna spektral. Euler dapat menggambar ketinggian fungsi pada plot dengan bayangan. Pada semua plot 3D, Euler dapat menghasilkan anaglyph merah/cyan.

\begin{itemize}
\item
  Rona: Mengaktifkan bayangan cahaya, bukan kabel.
\item
  \textgreater contour: Memplot garis kontur otomatis pada plot.
\item
  level=\ldots{} (atau level): Vektor nilai untuk garis kontur.
\end{itemize}

Defaultnya adalah level=``auto'', yang menghitung beberapa garis level secara otomatis. Seperti yang Anda lihat di plot, level sebenarnya adalah rentang level.

Gaya default dapat diubah. Untuk plot kontur berikut ini, kami menggunakan grid yang lebih halus untuk 100x100 titik, skala fungsi dan plot, dan menggunakan sudut pandang yang berbeda.

\textgreater plot3d(``exp(-x\textsuperscript{2-y}2)'',r=2,n=100,level=``thin'', \ldots{}\\
\textgreater{} \textgreater contour,\textgreater spectral,fscale=1,scale=1.1,angle=45°,height=20°):

\begin{figure}
\centering
\pandocbounded{\includegraphics[keepaspectratio]{images/APLIKOM\#3D_Sabilla Hanifah_23030630012-016.png}}
\caption{images/APLIKOM\#3D\_Sabilla\%20Hanifah\_23030630012-016.png}
\end{figure}

\textgreater plot3d(``exp(x*y)'',angle=100°,\textgreater contour,color=green):

\begin{figure}
\centering
\pandocbounded{\includegraphics[keepaspectratio]{images/APLIKOM\#3D_Sabilla Hanifah_23030630012-017.png}}
\caption{images/APLIKOM\#3D\_Sabilla\%20Hanifah\_23030630012-017.png}
\end{figure}

Bayangan default menggunakan warna abu-abu. Tetapi, kisaran warna spektral juga tersedia.

\begin{itemize}
\item
  \textgreater spektral: Menggunakan skema spektral default
\item
  color =\ldots: Menggunakan warna khusus atau skema spektral
\end{itemize}

Untuk plot berikut ini, kami menggunakan skema spektral default dan menambah jumlah titik untuk mendapatkan tampilan yang sangat halus.

\textgreater plot3d(``x\textsuperscript{2+y}2'',\textgreater spectral,\textgreater contour,n=100):

\begin{figure}
\centering
\pandocbounded{\includegraphics[keepaspectratio]{images/APLIKOM\#3D_Sabilla Hanifah_23030630012-018.png}}
\caption{images/APLIKOM\#3D\_Sabilla\%20Hanifah\_23030630012-018.png}
\end{figure}

Alih-alih garis level otomatis, kita juga dapat menetapkan nilai garis level. Hal ini akan menghasilkan garis level yang tipis, alih-alih rentang level.

\textgreater plot3d(``x\textsuperscript{2-y}2'',0,5,0,5,level=-1:0.1:1,color=redgreen):

\begin{figure}
\centering
\pandocbounded{\includegraphics[keepaspectratio]{images/APLIKOM\#3D_Sabilla Hanifah_23030630012-019.png}}
\caption{images/APLIKOM\#3D\_Sabilla\%20Hanifah\_23030630012-019.png}
\end{figure}

Pada plot berikut ini, kami menggunakan dua pita level yang sangat luas dari -0,1 hingga 1, dan dari 0,9 hingga 1. Ini dimasukkan sebagai matriks dengan batas-batas level sebagai kolom.

Selain itu, kami menghamparkan grid dengan 10 interval di setiap arah.

\textgreater plot3d(``x\textsuperscript{2+y}3'',level={[}-0.1,0.9;0,1{]}, \ldots{}\\
\textgreater{} \textgreater spectral,angle=30°,grid=10,contourcolor=gray):

\begin{figure}
\centering
\pandocbounded{\includegraphics[keepaspectratio]{images/APLIKOM\#3D_Sabilla Hanifah_23030630012-020.png}}
\caption{images/APLIKOM\#3D\_Sabilla\%20Hanifah\_23030630012-020.png}
\end{figure}

Pada contoh berikut, kami memplot himpunan, di mana

\[f(x,y) = x^y-y^x = 0\]Kita menggunakan satu garis tipis untuk garis level.

\textgreater plot3d(``x\textsuperscript{y-y}x'',level=0,a=0,b=6,c=0,d=6,contourcolor=red,n=100):

\begin{figure}
\centering
\pandocbounded{\includegraphics[keepaspectratio]{images/APLIKOM\#3D_Sabilla Hanifah_23030630012-022.png}}
\caption{images/APLIKOM\#3D\_Sabilla\%20Hanifah\_23030630012-022.png}
\end{figure}

Dimungkinkan untuk menampilkan bidang kontur di bawah plot. Warna dan jarak ke plot dapat ditentukan.

\textgreater plot3d(``x\textsuperscript{2+y}4'',\textgreater cp,cpcolor=green,cpdelta=0.2):

\begin{figure}
\centering
\pandocbounded{\includegraphics[keepaspectratio]{images/APLIKOM\#3D_Sabilla Hanifah_23030630012-023.png}}
\caption{images/APLIKOM\#3D\_Sabilla\%20Hanifah\_23030630012-023.png}
\end{figure}

Berikut ini beberapa gaya lainnya. Kami selalu mematikan bingkai, dan menggunakan berbagai skema warna untuk plot dan kisi-kisi.

\textgreater figure(2,2); \ldots{}\\
\textgreater{} expr=``y\textsuperscript{3-x}2''; \ldots{}\\
\textgreater{} figure(1); \ldots{}\\
\textgreater{} plot3d(expr,\textless frame,\textgreater cp,cpcolor=spectral); \ldots{}\\
\textgreater{} figure(2); \ldots{}\\
\textgreater{} plot3d(expr,\textless frame,\textgreater spectral,grid=10,cp=2); \ldots{}\\
\textgreater{} figure(3); \ldots{}\\
\textgreater{} plot3d(expr,\textless frame,\textgreater contour,color=gray,nc=5,cp=3,cpcolor=greenred); \ldots{}\\
\textgreater{} figure(4); \ldots{}\\
\textgreater{} plot3d(expr,\textless frame,\textgreater hue,grid=10,\textgreater transparent,\textgreater cp,cpcolor=gray); \ldots{}\\
\textgreater{} figure(0):

\begin{figure}
\centering
\pandocbounded{\includegraphics[keepaspectratio]{images/APLIKOM\#3D_Sabilla Hanifah_23030630012-024.png}}
\caption{images/APLIKOM\#3D\_Sabilla\%20Hanifah\_23030630012-024.png}
\end{figure}

Ada beberapa skema spektral lainnya, yang diberi nomor dari 1 hingga 9. Tetapi Anda juga dapat menggunakan color=value, di mana value

\begin{itemize}
\item
  spektral: untuk rentang dari biru ke merah
\item
  putih: untuk rentang yang lebih redup
\item
  kuningbiru, unguhijau, biru-kuning, hijau-merah
\item
  biru-kuning, hijau-ungu, kuning-biru, merah-hijau
\end{itemize}

\textgreater figure(3,3); \ldots{}\\
\textgreater{} for i=1:9; \ldots{}\\
\textgreater{} figure(i); plot3d(``x\textsuperscript{2+y}2'',spectral=i,\textgreater contour,\textgreater cp,\textless frame,zoom=4); \ldots{}\\
\textgreater{} end; \ldots{}\\
\textgreater{} figure(0):

\begin{figure}
\centering
\pandocbounded{\includegraphics[keepaspectratio]{images/APLIKOM\#3D_Sabilla Hanifah_23030630012-025.png}}
\caption{images/APLIKOM\#3D\_Sabilla\%20Hanifah\_23030630012-025.png}
\end{figure}

Sumber cahaya dapat diubah dengan l dan tombol kursor selama interaksi pengguna. Ini juga dapat ditetapkan dengan parameter.

\begin{itemize}
\item
  light: arah cahaya
\item
  amb: cahaya sekitar antara 0 dan 1
\end{itemize}

Perhatikan, bahwa program ini tidak membuat perbedaan di antara sisi-sisi plot. Tidak ada bayangan. Untuk ini, Anda memerlukan Povray.

\textgreater plot3d(``-x\textsuperscript{2-y}2'', \ldots{}\\
\textgreater{} hue=true,light={[}0,1,1{]},amb=0,user=true, \ldots{}\\
\textgreater{} title=``Press l and cursor keys (return to exit)''):

\begin{figure}
\centering
\pandocbounded{\includegraphics[keepaspectratio]{images/APLIKOM\#3D_Sabilla Hanifah_23030630012-026.png}}
\caption{images/APLIKOM\#3D\_Sabilla\%20Hanifah\_23030630012-026.png}
\end{figure}

Parameter warna mengubah warna permukaan. Warna garis level juga dapat diubah.

\textgreater plot3d(``-x\textsuperscript{2-y}2'',color=rgb(0.2,0.2,0),hue=true,frame=false, \ldots{}\\
\textgreater{} zoom=3,contourcolor=red,level=-2:0.1:1,dl=0.01):

\begin{figure}
\centering
\pandocbounded{\includegraphics[keepaspectratio]{images/APLIKOM\#3D_Sabilla Hanifah_23030630012-027.png}}
\caption{images/APLIKOM\#3D\_Sabilla\%20Hanifah\_23030630012-027.png}
\end{figure}

Warna 0 memberikan efek pelangi yang istimewa.

\textgreater plot3d(``x\textsuperscript{2/(x}2+y\^{}2+1)'',color=0,hue=true,grid=10):

\begin{figure}
\centering
\pandocbounded{\includegraphics[keepaspectratio]{images/APLIKOM\#3D_Sabilla Hanifah_23030630012-028.png}}
\caption{images/APLIKOM\#3D\_Sabilla\%20Hanifah\_23030630012-028.png}
\end{figure}

Permukaannya juga bisa transparan.

\textgreater plot3d(``x\textsuperscript{2+y}2'',\textgreater transparent,grid=10,wirecolor=red):

\begin{figure}
\centering
\pandocbounded{\includegraphics[keepaspectratio]{images/APLIKOM\#3D_Sabilla Hanifah_23030630012-029.png}}
\caption{images/APLIKOM\#3D\_Sabilla\%20Hanifah\_23030630012-029.png}
\end{figure}

\chapter{Plot Implisit}\label{plot-implisit}

Ada juga plot implisit dalam tiga dimensi. Euler menghasilkan potongan melalui objek. Fitur plot3d termasuk plot implisit. Plot-plot ini menunjukkan himpunan nol dari sebuah fungsi dalam tiga variabel. Solusi dari

dapat divisualisasikan dalam potongan yang sejajar dengan bidang x-y, bidang x-z, dan bidang y-z.

\begin{itemize}
\item
  implisit = 1: memotong sejajar dengan bidang y-z
\item
  implisit = 2: memotong sejajar dengan bidang x-z
\item
  implisit = 4: memotong sejajar dengan bidang x-y
\end{itemize}

Tambahkan nilai-nilai ini, jika Anda mau. Pada contoh, kami memplot

\textgreater plot3d(``x\textsuperscript{2+y}3+z*y-1'',r=5,implicit=3):

\begin{figure}
\centering
\pandocbounded{\includegraphics[keepaspectratio]{images/APLIKOM\#3D_Sabilla Hanifah_23030630012-030.png}}
\caption{images/APLIKOM\#3D\_Sabilla\%20Hanifah\_23030630012-030.png}
\end{figure}

\textgreater c=1; d=1;

\textgreater plot3d(``((x\textsuperscript{2+y}2-c\textsuperscript{2)}2+(z\textsuperscript{2-1)}2)*((y\textsuperscript{2+z}2-c\textsuperscript{2)}2+(x\textsuperscript{2-1)}2)*((z\textsuperscript{2+x}2-c\textsuperscript{2)}2+(y\textsuperscript{2-1)}2)-d'',r=2,\textless frame,\textgreater implicit,\textgreater user):

\begin{figure}
\centering
\pandocbounded{\includegraphics[keepaspectratio]{images/APLIKOM\#3D_Sabilla Hanifah_23030630012-031.png}}
\caption{images/APLIKOM\#3D\_Sabilla\%20Hanifah\_23030630012-031.png}
\end{figure}

\textgreater plot3d(``x\textsuperscript{2+y}2+4*x*z+z\^{}3'',\textgreater implicit,r=2,zoom=2.5):

\begin{figure}
\centering
\pandocbounded{\includegraphics[keepaspectratio]{images/APLIKOM\#3D_Sabilla Hanifah_23030630012-032.png}}
\caption{images/APLIKOM\#3D\_Sabilla\%20Hanifah\_23030630012-032.png}
\end{figure}

\chapter{Memplot Data 3D}\label{memplot-data-3d}

Sama seperti plot2d, plot3d menerima data. Untuk objek 3D, Anda perlu menyediakan matriks nilai x, y, dan z, atau tiga fungsi atau ekspresi fx(x,y), fy(x,y), fz(x,y).

\[\gamma(t,s) = (x(t,s),y(t,s),z(t,s))\]Karena x,y,z adalah matriks, kita mengasumsikan bahwa (t,s) berjalan melalui kotak persegi. Hasilnya, Anda dapat memplot gambar persegi panjang dalam ruang.

Anda dapat menggunakan bahasa matriks Euler untuk menghasilkan koordinat secara efektif.

Pada contoh berikut, kita menggunakan vektor nilai t dan vektor kolom nilai s untuk memparameterkan permukaan bola. Pada gambar kita dapat menandai daerah, dalam kasus kita daerah kutub.

\textgreater t=linspace(0,2pi,180); s=linspace(-pi/2,pi/2,90)'; \ldots{}\\
\textgreater{} x=cos(s)*cos(t); y=cos(s)*sin(t); z=sin(s); \ldots{}\\
\textgreater{} plot3d(x,y,z,\textgreater hue, \ldots{}\\
\textgreater{} color=blue,\textless frame,grid={[}10,20{]}, \ldots{}\\
\textgreater{} values=s,contourcolor=red,level={[}90°-24°;90°-22°{]}, \ldots{}\\
\textgreater{} scale=1.4,height=50°):

\begin{figure}
\centering
\pandocbounded{\includegraphics[keepaspectratio]{images/APLIKOM\#3D_Sabilla Hanifah_23030630012-034.png}}
\caption{images/APLIKOM\#3D\_Sabilla\%20Hanifah\_23030630012-034.png}
\end{figure}

Berikut ini adalah contoh, yang merupakan grafik suatu fungsi.

\textgreater t=-1:0.1:1; s=(-1:0.1:1)'; plot3d(t,s,t*s,grid=10):

\begin{figure}
\centering
\pandocbounded{\includegraphics[keepaspectratio]{images/APLIKOM\#3D_Sabilla Hanifah_23030630012-035.png}}
\caption{images/APLIKOM\#3D\_Sabilla\%20Hanifah\_23030630012-035.png}
\end{figure}

Namun demikian, kita bisa membuat segala macam permukaan. Berikut ini adalah permukaan yang sama dengan fungsi

\[x = y \, z\]\textgreater plot3d(t*s,t,s,angle=180°,grid=10):

\begin{figure}
\centering
\pandocbounded{\includegraphics[keepaspectratio]{images/APLIKOM\#3D_Sabilla Hanifah_23030630012-037.png}}
\caption{images/APLIKOM\#3D\_Sabilla\%20Hanifah\_23030630012-037.png}
\end{figure}

Dengan lebih banyak upaya, kita bisa menghasilkan banyak permukaan.

Dalam contoh berikut ini, kami membuat tampilan berbayang dari bola yang terdistorsi. Koordinat yang biasa digunakan untuk bola adalah

\[\gamma(t,s) = (\cos(t)\cos(s),\sin(t)\sin(s),\cos(s))\]dengan

\[0 \le t \le 2\pi, \quad \frac{-\pi}{2} \le s \le \frac{\pi}{2}.\]Kami mengubahnya dengan sebuah faktor

\[d(t,s) = \frac{\cos(4t)+\cos(8s)}{4}.\]\textgreater t=linspace(0,2pi,320); s=linspace(-pi/2,pi/2,160)'; \ldots{}\\
\textgreater{} d=1+0.2*(cos(4*t)+cos(8*s)); \ldots{}\\
\textgreater{} plot3d(cos(t)*cos(s)*d,sin(t)*cos(s)*d,sin(s)*d,hue=1, \ldots{}\\
\textgreater{} light={[}1,0,1{]},frame=0,zoom=5):

\begin{figure}
\centering
\pandocbounded{\includegraphics[keepaspectratio]{images/APLIKOM\#3D_Sabilla Hanifah_23030630012-041.png}}
\caption{images/APLIKOM\#3D\_Sabilla\%20Hanifah\_23030630012-041.png}
\end{figure}

Tentu saja, awan titik juga dimungkinkan. Untuk memplot data titik dalam ruang, kita membutuhkan tiga vektor untuk koordinat titik.

Gaya-gayanya sama seperti pada plot2d dengan points=true;

\textgreater n=500; \ldots{}\\
\textgreater{} plot3d(normal(1,n),normal(1,n),normal(1,n),points=true,style=``.''):

\begin{figure}
\centering
\pandocbounded{\includegraphics[keepaspectratio]{images/APLIKOM\#3D_Sabilla Hanifah_23030630012-042.png}}
\caption{images/APLIKOM\#3D\_Sabilla\%20Hanifah\_23030630012-042.png}
\end{figure}

Anda juga dapat memplot kurva dalam bentuk 3D. Dalam hal ini, akan lebih mudah untuk menghitung titik-titik kurva. Untuk kurva pada bidang, kami menggunakan urutan koordinat dan parameter wire = true.

\textgreater t=linspace(0,8pi,500); \ldots{}\\
\textgreater{} plot3d(sin(t),cos(t),t/10,\textgreater wire,zoom=3):

\begin{figure}
\centering
\pandocbounded{\includegraphics[keepaspectratio]{images/APLIKOM\#3D_Sabilla Hanifah_23030630012-043.png}}
\caption{images/APLIKOM\#3D\_Sabilla\%20Hanifah\_23030630012-043.png}
\end{figure}

\textgreater t=linspace(0,4pi,1000); plot3d(cos(t),sin(t),t/2pi,\textgreater wire, \ldots{}\\
\textgreater{} linewidth=3,wirecolor=blue):

\begin{figure}
\centering
\pandocbounded{\includegraphics[keepaspectratio]{images/APLIKOM\#3D_Sabilla Hanifah_23030630012-044.png}}
\caption{images/APLIKOM\#3D\_Sabilla\%20Hanifah\_23030630012-044.png}
\end{figure}

\textgreater X=cumsum(normal(3,100)); \ldots{}\\
\textgreater{} plot3d(X{[}1{]},X{[}2{]},X{[}3{]},\textgreater anaglyph,\textgreater wire):

\begin{figure}
\centering
\pandocbounded{\includegraphics[keepaspectratio]{images/APLIKOM\#3D_Sabilla Hanifah_23030630012-045.png}}
\caption{images/APLIKOM\#3D\_Sabilla\%20Hanifah\_23030630012-045.png}
\end{figure}

EMT juga dapat membuat plot dalam mode anaglyph. Untuk melihat plot semacam itu, Anda memerlukan kacamata merah/cyan.

\textgreater{} plot3d(``x\textsuperscript{2+y}3'',\textgreater anaglyph,\textgreater contour,angle=30°):

\begin{figure}
\centering
\pandocbounded{\includegraphics[keepaspectratio]{images/APLIKOM\#3D_Sabilla Hanifah_23030630012-046.png}}
\caption{images/APLIKOM\#3D\_Sabilla\%20Hanifah\_23030630012-046.png}
\end{figure}

Sering kali, skema warna spektral digunakan untuk plot. Hal ini menekankan ketinggian fungsi.

\textgreater plot3d(``x\textsuperscript{2*y}3-y'',\textgreater spectral,\textgreater contour,zoom=3.2):

\begin{figure}
\centering
\pandocbounded{\includegraphics[keepaspectratio]{images/APLIKOM\#3D_Sabilla Hanifah_23030630012-047.png}}
\caption{images/APLIKOM\#3D\_Sabilla\%20Hanifah\_23030630012-047.png}
\end{figure}

Euler juga dapat memplot permukaan yang diparameterkan, ketika parameternya adalah nilai x, y, dan z dari gambar kisi-kisi persegi panjang di dalam ruang.

Untuk demo berikut ini, kami menyiapkan parameter u dan v, dan menghasilkan koordinat ruang dari parameter ini.

\textgreater u=linspace(-1,1,10); v=linspace(0,2*pi,50)'; \ldots{}\\
\textgreater{} X=(3+u*cos(v/2))*cos(v); Y=(3+u*cos(v/2))*sin(v); Z=u*sin(v/2); \ldots{}\\
\textgreater{} plot3d(X,Y,Z,\textgreater anaglyph,\textless frame,\textgreater wire,scale=2.3):

\begin{figure}
\centering
\pandocbounded{\includegraphics[keepaspectratio]{images/APLIKOM\#3D_Sabilla Hanifah_23030630012-048.png}}
\caption{images/APLIKOM\#3D\_Sabilla\%20Hanifah\_23030630012-048.png}
\end{figure}

Berikut ini contoh yang lebih rumit, yang tampak megah dengan kacamata merah/cyan.

\textgreater u:=linspace(-pi,pi,160); v:=linspace(-pi,pi,400)'; \ldots{}\\
\textgreater{} x:=(4*(1+.25*sin(3*v))+cos(u))*cos(2*v); \ldots{}\\
\textgreater{} y:=(4*(1+.25*sin(3*v))+cos(u))*sin(2*v); \ldots{}\\
\textgreater{} z=sin(u)+2*cos(3*v); \ldots{}\\
\textgreater{} plot3d(x,y,z,frame=0,scale=1.5,hue=1,light={[}1,0,-1{]},zoom=2.8,\textgreater anaglyph):

\begin{figure}
\centering
\pandocbounded{\includegraphics[keepaspectratio]{images/APLIKOM\#3D_Sabilla Hanifah_23030630012-049.png}}
\caption{images/APLIKOM\#3D\_Sabilla\%20Hanifah\_23030630012-049.png}
\end{figure}

\chapter{Plot Statistik}\label{plot-statistik}

Bar plot atau plot batang juga dapat digunakan. Untuk ini, kita harus menyediakan

\begin{itemize}
\item
  x: vektor baris dengan n+1 elemen
\item
  y: vektor kolom dengan n+1 elemen
\item
  z: matriks nilai nxn.
\end{itemize}

z dapat lebih besar, tetapi hanya nilai nxn yang akan digunakan.

Pada contoh, pertama-tama kita menghitung nilainya. Kemudian kita sesuaikan x dan y, sehingga vektor-vektor tersebut berada di tengah-tengah nilai yang digunakan.

\textgreater x=-1:0.1:1; y=x'; z=x\textsuperscript{2+y}2; \ldots{}\\
\textgreater{} xa=(x\textbar1.1)-0.05; ya=(y\_1.1)-0.05; \ldots{}\\
\textgreater{} plot3d(xa,ya,z,bar=true):

\begin{figure}
\centering
\pandocbounded{\includegraphics[keepaspectratio]{images/APLIKOM\#3D_Sabilla Hanifah_23030630012-050.png}}
\caption{images/APLIKOM\#3D\_Sabilla\%20Hanifah\_23030630012-050.png}
\end{figure}

Hal ini memungkinkan untuk membagi plot permukaan menjadi dua bagian atau lebih.

\textgreater x=-1:0.1:1; y=x'; z=x+y; d=zeros(size(x)); \ldots{}\\
\textgreater{} plot3d(x,y,z,disconnect=2:2:20):

\begin{figure}
\centering
\pandocbounded{\includegraphics[keepaspectratio]{images/APLIKOM\#3D_Sabilla Hanifah_23030630012-051.png}}
\caption{images/APLIKOM\#3D\_Sabilla\%20Hanifah\_23030630012-051.png}
\end{figure}

Jika memuat atau menghasilkan matriks data M dari file dan perlu memplotnya dalam 3D, Anda dapat menskalakan matriks ke {[}-1,1{]} dengan scale(M), atau menskalakan matriks dengan \textgreater zscale. Hal ini dapat dikombinasikan dengan faktor penskalaan individual yang diterapkan sebagai tambahan.

\textgreater i=1:20; j=i'; \ldots{}\\
\textgreater{} plot3d(i*j\^{}2+100*normal(20,20),\textgreater zscale,scale={[}1,1,1.5{]},angle=-40°,zoom=1.8):

\begin{figure}
\centering
\pandocbounded{\includegraphics[keepaspectratio]{images/APLIKOM\#3D_Sabilla Hanifah_23030630012-052.png}}
\caption{images/APLIKOM\#3D\_Sabilla\%20Hanifah\_23030630012-052.png}
\end{figure}

\textgreater Z=intrandom(5,100,6); v=zeros(5,6); \ldots{}\\
\textgreater{} loop 1 to 5; v{[}\#{]}=getmultiplicities(1:6,Z{[}\#{]}); end; \ldots{}\\
\textgreater{} columnsplot3d(v',scols=1:5,ccols={[}1:5{]}):

\begin{figure}
\centering
\pandocbounded{\includegraphics[keepaspectratio]{images/APLIKOM\#3D_Sabilla Hanifah_23030630012-053.png}}
\caption{images/APLIKOM\#3D\_Sabilla\%20Hanifah\_23030630012-053.png}
\end{figure}

\chapter{Permukaan Benda Putar}\label{permukaan-benda-putar}

\textgreater plot2d(``(x\textsuperscript{2+y}2-1)\textsuperscript{3-x}2*y\^{}3'',r=1.3, \ldots{}\\
\textgreater{} style=``\#'',color=red,\textless outline, \ldots{}\\
\textgreater{} level={[}-2;0{]},n=100):

\begin{figure}
\centering
\pandocbounded{\includegraphics[keepaspectratio]{images/APLIKOM\#3D_Sabilla Hanifah_23030630012-054.png}}
\caption{images/APLIKOM\#3D\_Sabilla\%20Hanifah\_23030630012-054.png}
\end{figure}

\textgreater ekspresi \&= (x\textsuperscript{2+y}2-1)\textsuperscript{3-x}2*y\^{}3; \$ekspresi

\[\left(y^2+x^2-1\right)^3-x^2\,y^3\]Kami ingin memutar kurva hati di sekitar sumbu y. Berikut ini adalah ekspresi yang mendefinisikan hati:

\[f(x,y)=(x^2+y^2-1)^3-x^2.y^3.\]Selanjutnya kita atur

\[x = r.cos(a), \ kuad y = r.sin(a).\]\textgreater function fr(r,a) \&= ekspresi with {[}x=r*cos(a),y=r*sin(a){]} \textbar{} trigreduce; \$fr(r,a)

\[\left(r^2-1\right)^3+\frac{\left(\sin \left(5\,a\right)-\sin \left(
 3\,a\right)-2\,\sin a\right)\,r^5}{16}\]Hal ini memungkinkan untuk mendefinisikan fungsi numerik, yang menyelesaikan untuk r, jika a diberikan. Dengan fungsi tersebut kita dapat memplotkan hati yang diputar sebagai permukaan parametrik.

\textgreater function map f(a) := bisect(``fr'',0,2;a); \ldots{}\\
\textgreater{} t=linspace(-pi/2,pi/2,100); r=f(t); \ldots{}\\
\textgreater{} s=linspace(pi,2pi,100)'; \ldots{}\\
\textgreater{} plot3d(r*cos(t)*sin(s),r*cos(t)*cos(s),r*sin(t), \ldots{}\\
\textgreater{} \textgreater hue,\textless frame,color=red,zoom=4,amb=0,max=0.7,grid=12,height=50°):

\begin{figure}
\centering
\pandocbounded{\includegraphics[keepaspectratio]{images/APLIKOM\#3D_Sabilla Hanifah_23030630012-059.png}}
\caption{images/APLIKOM\#3D\_Sabilla\%20Hanifah\_23030630012-059.png}
\end{figure}

Berikut ini adalah plot 3D dari gambar di atas yang diputar mengelilingi sumbu-z. Kami mendefinisikan fungsi, yang menggambarkan objek.

\textgreater function f(x,y,z) \ldots{}

\begin{verbatim}
r=x^2+y^2;
return (r+z^2-1)^3-r*z^3;
 endfunction
\end{verbatim}

\textgreater plot3d(``f(x,y,z)'', \ldots{}\\
\textgreater{} xmin=0,xmax=1.2,ymin=-1.2,ymax=1.2,zmin=-1.2,zmax=1.4, \ldots{}\\
\textgreater{} implicit=1,angle=-30°,zoom=2.5,n={[}10,100,60{]},\textgreater anaglyph):

\begin{figure}
\centering
\pandocbounded{\includegraphics[keepaspectratio]{images/APLIKOM\#3D_Sabilla Hanifah_23030630012-060.png}}
\caption{images/APLIKOM\#3D\_Sabilla\%20Hanifah\_23030630012-060.png}
\end{figure}

\chapter{Plot 3D Khusus}\label{plot-3d-khusus}

Fungsi plot3d memang bagus untuk dimiliki, tetapi tidak memenuhi semua kebutuhan. Selain rutinitas yang lebih mendasar, Anda juga bisa mendapatkan plot berbingkai dari objek apa pun yang Anda sukai.

Meskipun Euler bukan program 3D, namun dapat menggabungkan beberapa objek dasar. Kami mencoba memvisualisasikan parabola dan garis singgungnya.

\textgreater function myplot \ldots{}

\begin{verbatim}
  y=-1:0.01:1; x=(-1:0.01:1)';
  plot3d(x,y,0.2*(x-0.1)/2,<scale,<frame,>hue, ..
    hues=0.5,>contour,color=orange);
  h=holding(1);
  plot3d(x,y,(x^2+y^2)/2,<scale,<frame,>contour,>hue);
  holding(h);
endfunction
\end{verbatim}

Sekarang framedplot() menyediakan frame, dan mengatur tampilan.

\textgreater framedplot(``myplot'',{[}-1,1,-1,1,0,1{]},height=0,angle=-30°, \ldots{}\\
\textgreater{} center={[}0,0,-0.7{]},zoom=3):

\begin{figure}
\centering
\pandocbounded{\includegraphics[keepaspectratio]{images/APLIKOM\#3D_Sabilla Hanifah_23030630012-061.png}}
\caption{images/APLIKOM\#3D\_Sabilla\%20Hanifah\_23030630012-061.png}
\end{figure}

Dengan cara yang sama, Anda dapat memplot bidang kontur secara manual. Perhatikan bahwa plot3d() mengatur jendela ke fullwindow() secara default, namun plotcontourplane() mengasumsikannya.

\textgreater x=-1:0.02:1.1; y=x'; z=x\textsuperscript{2-y}4;

\textgreater function myplot (x,y,z) \ldots{}\\
\textgreater{}\\

\textgreater myplot(x,y,z):

\begin{figure}
\centering
\pandocbounded{\includegraphics[keepaspectratio]{images/APLIKOM\#3D_Sabilla Hanifah_23030630012-062.png}}
\caption{images/APLIKOM\#3D\_Sabilla\%20Hanifah\_23030630012-062.png}
\end{figure}

\chapter{Animasi}\label{animasi}

Euler dapat menggunakan frame untuk melakukan pra-komputasi animasi.

Salah satu fungsi yang memanfaatkan teknik ini adalah rotate. Fungsi ini dapat mengubah sudut pandang dan menggambar ulang plot 3D. Fungsi ini memanggil addpage() untuk setiap plot baru. Akhirnya fungsi ini menganimasikan plot tersebut.

Silakan pelajari sumber dari rotate untuk melihat lebih detail.

\textgreater function testplot () := plot3d(``x\textsuperscript{2+y}3''); \ldots{}\\
\textgreater{} rotate(``testplot''); testplot():

\begin{figure}
\centering
\pandocbounded{\includegraphics[keepaspectratio]{images/APLIKOM\#3D_Sabilla Hanifah_23030630012-063.png}}
\caption{images/APLIKOM\#3D\_Sabilla\%20Hanifah\_23030630012-063.png}
\end{figure}

\chapter{Menggambar Povray}\label{menggambar-povray}

Dengan bantuan file Euler povray.e, Euler dapat menghasilkan file Povray. Hasilnya sangat bagus untuk dilihat.

Anda perlu menginstal Povray (32bit atau 64bit) dari \textless a href=``http://www.povray.org/, dan meletakkan sub-direktori''bin'' dari Povray ke dalam jalur lingkungan, atau mengatur variabel ``defaultpovray'' dengan jalur lengkap yang mengarah ke ``pvengine.exe''.''\textgreater http://www.povray.org/, dan meletakkan sub-direktori ``bin'' dari Povray ke dalam jalur lingkungan, atau mengatur variabel ``defaultpovray'' dengan jalur lengkap yang mengarah ke ``pvengine.exe''.

Antarmuka Povray dari Euler menghasilkan file Povray di direktori home pengguna, dan memanggil Povray untuk mengurai file-file ini. Nama file default adalah current.pov, dan direktori defaultnya adalah eulerhome(), biasanya c:\Users\Username\Euler. Povray menghasilkan sebuah file PNG, yang dapat dimuat oleh Euler ke dalam notebook. Untuk membersihkan berkas-berkas ini, gunakan povclear().

Fungsi pov3d memiliki semangat yang sama dengan plot3d. Fungsi ini dapat menghasilkan grafik dari sebuah fungsi f(x,y), atau sebuah permukaan dengan koordinat X,Y,Z dalam bentuk matriks, termasuk garis-garis level yang bersifat opsional. Fungsi ini memulai raytracer secara otomatis, dan memuat adegan ke dalam notebook Euler.

Selain pov3d(), ada banyak fungsi yang menghasilkan objek Povray. Fungsi-fungsi ini mengembalikan string, yang berisi kode Povray untuk objek. Untuk menggunakan fungsi-fungsi ini, mulai file Povray dengan povstart(). Kemudian gunakan writeln(\ldots) untuk menulis objek ke file scene. Terakhir, akhiri file dengan povend(). Secara default, raytracer akan dimulai, dan PNG akan dimasukkan ke dalam buku catatan Euler.

Fungsi objek memiliki parameter yang disebut ``look'', yang membutuhkan string dengan kode povray untuk tekstur dan hasil akhir objek. Fungsi povlook() dapat digunakan untuk menghasilkan string ini. Fungsi ini memiliki parameter untuk warna, transparansi, Phong Shading, dll.

Perhatikan bahwa Povray universe memiliki sistem koordinat lain. Antarmuka ini menerjemahkan semua koordinat ke sistem Povray. Jadi Anda bisa tetap berpikir dalam sistem koordinat Euler dengan z menunjuk vertikal ke atas, dan sumbu x, y, z di tangan kanan. Anda perlu memuat file povray.

\textgreater load povray;

Pastikan direktori bin povray berada di dalam path. Jika tidak, edit variabel berikut sehingga berisi jalur ke povray yang dapat dieksekusi.

\textgreater defaultpovray=``C:\textbackslash Program Files\textbackslash POV-Ray\textbackslash v3.7\textbackslash bin\textbackslash pvengine.exe''

\begin{verbatim}
C:\Program Files\POV-Ray\v3.7\bin\pvengine.exe
\end{verbatim}

Untuk kesan pertama, kita plot sebuah fungsi sederhana. Perintah berikut ini menghasilkan file povray di direktori pengguna Anda, dan menjalankan Povray untuk melacak sinar pada file ini.

Jika Anda memulai perintah berikut, GUI Povray akan terbuka, menjalankan file, dan menutup secara otomatis. Karena alasan keamanan, Anda akan ditanya, apakah Anda ingin mengizinkan file exe dijalankan. Anda dapat menekan cancel untuk menghentikan pertanyaan lebih lanjut. Anda mungkin harus menekan OK pada jendela Povray untuk mengetahui dialog awal Povray.

\textgreater plot3d(``x\textsuperscript{2+y}2'',zoom=2):

\begin{figure}
\centering
\pandocbounded{\includegraphics[keepaspectratio]{images/APLIKOM\#3D_Sabilla Hanifah_23030630012-064.png}}
\caption{images/APLIKOM\#3D\_Sabilla\%20Hanifah\_23030630012-064.png}
\end{figure}

\textgreater load povray;

\textgreater defaultpovray=``C:\textbackslash Program Files\textbackslash POV-Ray\textbackslash v3.7\textbackslash bin\textbackslash pvengine.exe''

\begin{verbatim}
C:\Program Files\POV-Ray\v3.7\bin\pvengine.exe
\end{verbatim}

\textgreater pov3d(``x\textsuperscript{2+y}2'',zoom=3);

\begin{figure}
\centering
\pandocbounded{\includegraphics[keepaspectratio]{images/APLIKOM\#3D_Sabilla Hanifah_23030630012-065.png}}
\caption{images/APLIKOM\#3D\_Sabilla\%20Hanifah\_23030630012-065.png}
\end{figure}

Kita dapat membuat fungsi menjadi transparan dan menambahkan hasil akhir lainnya. Kita juga dapat menambahkan garis level ke plot fungsi.

\textgreater pov3d(``x\textsuperscript{2+y}3'',axiscolor=red,angle=-45°,\textgreater anaglyph, \ldots{}\\
\textgreater{} look=povlook(cyan,0.2),level=-1:0.5:1,zoom=3.8);

\begin{figure}
\centering
\pandocbounded{\includegraphics[keepaspectratio]{images/APLIKOM\#3D_Sabilla Hanifah_23030630012-066.png}}
\caption{images/APLIKOM\#3D\_Sabilla\%20Hanifah\_23030630012-066.png}
\end{figure}

Kadang-kadang perlu untuk mencegah penskalaan fungsi, dan menskalakan fungsi dengan tangan.

Kami memplot kumpulan titik pada bidang kompleks, di mana hasil kali jarak ke 1 dan -1 sama dengan 1.

\textgreater pov3d(``((x-1)\textsuperscript{2+y}2)*((x+1)\textsuperscript{2+y}2)/40'',r=2, \ldots{}\\
\textgreater{} angle=-120°,level=1/40,dlevel=0.005,light={[}-1,1,1{]},height=10°,n=50, \ldots{}\\
\textgreater{} \textless fscale,zoom=3.8);

\begin{figure}
\centering
\pandocbounded{\includegraphics[keepaspectratio]{images/APLIKOM\#3D_Sabilla Hanifah_23030630012-067.png}}
\caption{images/APLIKOM\#3D\_Sabilla\%20Hanifah\_23030630012-067.png}
\end{figure}

\chapter{Merencanakan dengan Koordinat}\label{merencanakan-dengan-koordinat}

Sebagai pengganti fungsi, kita dapat membuat plot dengan koordinat. Seperti pada plot3d, kita membutuhkan tiga matriks untuk mendefinisikan objek.

Pada contoh, kita memutar sebuah fungsi pada sumbu z.

\textgreater function f(x) := x\^{}3-x+1; \ldots{}\\
\textgreater{} x=-1:0.01:1; t=linspace(0,2pi,50)'; \ldots{}\\
\textgreater{} Z=x; X=cos(t)*f(x); Y=sin(t)*f(x); \ldots{}\\
\textgreater{} pov3d(X,Y,Z,angle=40°,look=povlook(red,0.1),height=50°,axis=0,zoom=4,light={[}10,5,15{]});

\begin{figure}
\centering
\pandocbounded{\includegraphics[keepaspectratio]{images/APLIKOM\#3D_Sabilla Hanifah_23030630012-068.png}}
\caption{images/APLIKOM\#3D\_Sabilla\%20Hanifah\_23030630012-068.png}
\end{figure}

Pada contoh berikut, kita memplot gelombang teredam. Kami menghasilkan gelombang dengan bahasa matriks Euler.

Kami juga menunjukkan, bagaimana objek tambahan dapat ditambahkan ke adegan pov3d. Untuk pembuatan objek, lihat contoh berikut. Perhatikan bahwa plot3d menskalakan plot, sehingga sesuai dengan kubus satuan.

\textgreater r=linspace(0,1,80); phi=linspace(0,2pi,80)'; \ldots{}\\
\textgreater{} x=r*cos(phi); y=r*sin(phi); z=exp(-5*r)*cos(8*pi*r)/3; \ldots{}\\
\textgreater{} pov3d(x,y,z,zoom=6,axis=0,height=30°,add=povsphere({[}0.5,0,0.25{]},0.15,povlook(red)), \ldots{}\\
\textgreater{} w=500,h=300);

\begin{figure}
\centering
\pandocbounded{\includegraphics[keepaspectratio]{images/APLIKOM\#3D_Sabilla Hanifah_23030630012-069.png}}
\caption{images/APLIKOM\#3D\_Sabilla\%20Hanifah\_23030630012-069.png}
\end{figure}

Dengan metode bayangan canggih Povray, hanya sedikit titik yang bisa menghasilkan permukaan yang sangat halus. Hanya pada batas-batas dan bayangan, trik ini bisa terlihat jelas.

Untuk itu, kita perlu menambahkan vektor normal di setiap titik matriks.

\textgreater Z \&= x\textsuperscript{2*y}3

\begin{verbatim}
                                 2  3
                                x  y
\end{verbatim}

Persamaan permukaannya adalah {[}x,y,Z{]}. Kami menghitung dua turunan terhadap x dan y dari persamaan ini dan mengambil hasil perkalian silang sebagai normal.

\textgreater dx \&= diff({[}x,y,Z{]},x); dy \&= diff({[}x,y,Z{]},y);

Kami mendefinisikan normal sebagai hasil kali silang dari turunan ini, dan mendefinisikan fungsi koordinat.

\textgreater N \&= crossproduct(dx,dy); NX \&= N{[}1{]}; NY \&= N{[}2{]}; NZ \&= N{[}3{]}; N,

\begin{verbatim}
                               3       2  2
                       [- 2 x y , - 3 x  y , 1]
\end{verbatim}

Kami hanya menggunakan 25 poin.

\textgreater x=-1:0.5:1; y=x';

\textgreater pov3d(x,y,Z(x,y),angle=10°, \ldots{}\\
\textgreater{} xv=NX(x,y),yv=NY(x,y),zv=NZ(x,y),\textless shadow);

\begin{figure}
\centering
\pandocbounded{\includegraphics[keepaspectratio]{images/APLIKOM\#3D_Sabilla Hanifah_23030630012-070.png}}
\caption{images/APLIKOM\#3D\_Sabilla\%20Hanifah\_23030630012-070.png}
\end{figure}

Berikut ini adalah simpul Trefoil yang dibuat oleh A. Busser di Povray. Ada versi yang lebih baik dari ini dalam contoh.

Trefoil Knot

Untuk tampilan yang bagus dengan tidak terlalu banyak titik, kami menambahkan vektor normal di sini. Kami menggunakan Maxima untuk menghitung normal untuk kami. Pertama, tiga fungsi untuk koordinat sebagai ekspresi simbolis.

\textgreater X \&= ((4+sin(3*y))+cos(x))*cos(2*y); \ldots{}\\
\textgreater{} Y \&= ((4+sin(3*y))+cos(x))*sin(2*y); \ldots{}\\
\textgreater{} Z \&= sin(x)+2*cos(3*y);

Kemudian dua vektor turunan terhadap x dan y.

\textgreater dx \&= diff({[}X,Y,Z{]},x); dy \&= diff({[}X,Y,Z{]},y);

Sekarang yang normal, yang merupakan produk silang dari dua turunan.

\textgreater dn \&= crossproduct(dx,dy);

Kami sekarang mengevaluasi semua ini secara numerik.

\textgreater x:=linspace(-\%pi,\%pi,40); y:=linspace(-\%pi,\%pi,100)';

Vektor normal adalah evaluasi dari ekspresi simbolik dn{[}i{]} untuk i=1,2,3. Sintaks untuk ini adalah \&``ekspresi''(parameter). Ini adalah sebuah alternatif dari metode pada contoh sebelumnya, di mana kita mendefinisikan ekspresi simbolik NX, NY, NZ terlebih dahulu.

\textgreater pov3d(X(x,y),Y(x,y),Z(x,y),\textgreater anaglyph,axis=0,zoom=5,w=450,h=350, \ldots{}\\
\textgreater{} \textless shadow,look=povlook(blue), \ldots{}\\
\textgreater{} xv=\&``dn{[}1{]}''(x,y), yv=\&``dn{[}2{]}''(x,y), zv=\&``dn{[}3{]}''(x,y));

\begin{figure}
\centering
\pandocbounded{\includegraphics[keepaspectratio]{images/APLIKOM\#3D_Sabilla Hanifah_23030630012-071.png}}
\caption{images/APLIKOM\#3D\_Sabilla\%20Hanifah\_23030630012-071.png}
\end{figure}

Kami juga dapat menghasilkan kisi-kisi dalam bentuk 3D.

\textgreater povstart(zoom=4); \ldots{}\\
\textgreater{} x=-1:0.5:1; r=1-(x+1)\^{}2/6; \ldots{}\\
\textgreater{} t=(0°:30°:360°)'; y=r*cos(t); z=r*sin(t); \ldots{}\\
\textgreater{} writeln(povgrid(x,y,z,d=0.02,dballs=0.05)); \ldots{}\\
\textgreater{} povend();

\begin{figure}
\centering
\pandocbounded{\includegraphics[keepaspectratio]{images/APLIKOM\#3D_Sabilla Hanifah_23030630012-072.png}}
\caption{images/APLIKOM\#3D\_Sabilla\%20Hanifah\_23030630012-072.png}
\end{figure}

Dengan povgrid(), kurva dapat dibuat.

\textgreater povstart(center={[}0,0,1{]},zoom=3.6); \ldots{}\\
\textgreater{} t=linspace(0,2,1000); r=exp(-t); \ldots{}\\
\textgreater{} x=cos(2*pi*10*t)*r; y=sin(2*pi*10*t)*r; z=t; \ldots{}\\
\textgreater{} writeln(povgrid(x,y,z,povlook(red))); \ldots{}\\
\textgreater{} writeAxis(0,2,axis=3); \ldots{}\\
\textgreater{} povend();

\begin{figure}
\centering
\pandocbounded{\includegraphics[keepaspectratio]{images/APLIKOM\#3D_Sabilla Hanifah_23030630012-073.png}}
\caption{images/APLIKOM\#3D\_Sabilla\%20Hanifah\_23030630012-073.png}
\end{figure}

\chapter{Objek Povray}\label{objek-povray}

Di atas, kami menggunakan pov3d untuk memplot permukaan. Antarmuka povray di Euler juga dapat menghasilkan objek Povray. Objek-objek ini disimpan sebagai string di Euler, dan perlu ditulis ke file Povray.

Kita memulai output dengan povstart().

\textgreater povstart(zoom=4);

Pertama, kita mendefinisikan tiga silinder, dan menyimpannya dalam bentuk string di Euler.

Fungsi povx() dll. hanya mengembalikan vektor {[}1,0,0{]}, yang dapat digunakan sebagai gantinya.

\textgreater c1=povcylinder(-povx,povx,1,povlook(red)); \ldots{}\\
\textgreater{} c2=povcylinder(-povy,povy,1,povlook(yellow)); \ldots{}\\
\textgreater{} c3=povcylinder(-povz,povz,1,povlook(blue)); \ldots{}\\
\textgreater{}\\
String berisi kode Povray, yang tidak perlu kita pahami pada saat itu.

\textgreater c2

\begin{verbatim}
cylinder { &lt;0,0,-1&gt;, &lt;0,0,1&gt;, 1
 texture { pigment { color rgb &lt;0.941176,0.941176,0.392157&gt; }  } 
 finish { ambient 0.2 } 
 }
\end{verbatim}

Seperti yang Anda lihat, kami menambahkan tekstur ke objek dalam tiga warna berbeda.

Hal ini dilakukan dengan povlook(), yang mengembalikan sebuah string dengan kode Povray yang relevan. Kita dapat menggunakan warna default Euler, atau menentukan warna kita sendiri. Kita juga dapat menambahkan transparansi, atau mengubah cahaya sekitar.

\textgreater povlook(rgb(0.1,0.2,0.3),0.1,0.5)

\begin{verbatim}
 texture { pigment { color rgbf &lt;0.101961,0.2,0.301961,0.1&gt; }  } 
 finish { ambient 0.5 } 
\end{verbatim}

Sekarang kita mendefinisikan objek perpotongan, dan menulis hasilnya ke file.

\textgreater writeln(povintersection({[}c1,c2,c3{]}));

Perpotongan tiga silinder sulit untuk divisualisasikan, jika Anda tidak pernah melihatnya sebelumnya.

\textgreater povend;

\begin{figure}
\centering
\pandocbounded{\includegraphics[keepaspectratio]{images/APLIKOM\#3D_Sabilla Hanifah_23030630012-074.png}}
\caption{images/APLIKOM\#3D\_Sabilla\%20Hanifah\_23030630012-074.png}
\end{figure}

Fungsi-fungsi berikut ini menghasilkan fraktal secara rekursif.

Fungsi pertama menunjukkan, bagaimana Euler menangani objek Povray sederhana. Fungsi povbox() mengembalikan sebuah string, yang berisi koordinat kotak, tekstur dan hasil akhir.

\textgreater function onebox(x,y,z,d) := povbox({[}x,y,z{]},{[}x+d,y+d,z+d{]},povlook());

\textgreater function fractal (x,y,z,h,n) \ldots{}\\
\textgreater{}\\

\textgreater povstart(fade=10,\textless shadow);

\textgreater fractal(-1,-1,-1,2,4);

\textgreater povend();

\begin{figure}
\centering
\pandocbounded{\includegraphics[keepaspectratio]{images/APLIKOM\#3D_Sabilla Hanifah_23030630012-075.png}}
\caption{images/APLIKOM\#3D\_Sabilla\%20Hanifah\_23030630012-075.png}
\end{figure}

\textgreater load povray;

\textgreater defaultpovray=``C:\textbackslash Program Files\textbackslash POV-Ray\textbackslash v3.7\textbackslash bin\textbackslash pvengine.exe''

\begin{verbatim}
C:\Program Files\POV-Ray\v3.7\bin\pvengine.exe
\end{verbatim}

Perbedaan memungkinkan pemotongan satu objek dari objek lainnya. Seperti persimpangan, ada bagian dari objek CSG Povray.

\textgreater{} povstart(light={[}5,-5,5{]},fade=10);

Untuk demonstrasi ini, kita mendefinisikan sebuah objek di Povray, alih-alih menggunakan string di Euler. Definisi akan langsung dituliskan ke file.

Koordinat kotak -1 berarti {[}-1,-1,-1{]}.

\textgreater povdefine(``mycube'',povbox(-1,1));

Kita dapat menggunakan objek ini dalam povobject(), yang mengembalikan sebuah string seperti biasa.

\textgreater c1=povobject(``mycube'',povlook(red));

Kami menghasilkan kubus kedua, dan memutar serta menskalakannya sedikit.

\textgreater c2=povobject(``mycube'',povlook(yellow),translate={[}1,1,1{]}, \ldots{}\\
\textgreater{} rotate=xrotate(10°)+yrotate(10°), scale=1.2);

Kemudian kita ambil selisih dari kedua objek tersebut.

\textgreater writeln(povdifference(c1,c2));

Sekarang tambahkan tiga sumbu.

\textgreater writeAxis(-1.2,1.2,axis=1); \ldots{}\\
\textgreater{} writeAxis(-1.2,1.2,axis=2); \ldots{}\\
\textgreater{} writeAxis(-1.2,1.2,axis=4); \ldots{}\\
\textgreater{} povend();

\begin{figure}
\centering
\pandocbounded{\includegraphics[keepaspectratio]{images/APLIKOM\#3D_Sabilla Hanifah_23030630012-076.png}}
\caption{images/APLIKOM\#3D\_Sabilla\%20Hanifah\_23030630012-076.png}
\end{figure}

\textgreater load povray;

\textgreater defaultpovray=``C:\textbackslash Program Files\textbackslash POV-Ray\textbackslash v3.7\textbackslash bin\textbackslash pvengine.exe''

\begin{verbatim}
C:\Program Files\POV-Ray\v3.7\bin\pvengine.exe
\end{verbatim}

\chapter{Fungsi Implisit}\label{fungsi-implisit}

Povray dapat memplot himpunan di mana f(x,y,z)=0, seperti parameter implisit di plot3d. Namun, hasilnya terlihat jauh lebih baik.

Sintaks untuk fungsi-fungsi tersebut sedikit berbeda. Anda tidak dapat menggunakan output dari ekspresi Maxima atau Euler.

\[((x^2+y^2-c^2)^2+(z^2-1)^2)*((y^2+z^2-c^2)^2+(x^2-1)^2)*((z^2+x^2-c^2)^2+(y^2-1)^2)=d\]\textgreater c=0.1; d=0.1; \ldots{}\\
\textgreater{} writeln(povsurface(``(pow(pow(x,2)+pow(y,2)-pow(c,2),2)+pow(pow(z,2)-1,2))*(pow(pow(y,2)+pow(z,2)-pow(c,2),2)+pow(pow(x,2)-1,2))*(pow(pow(z,2)+pow(x,2)-pow(c,2),2)+pow(pow(y,2)-1,2))-d'',povlook(red))); \ldots{}\\
\textgreater{} povend();

\begin{verbatim}
object {
isosurface {
function { (pow(pow(x,2)+pow(y,2)-pow(c,2),2)+pow(pow(z,2)-1,2))*(pow(pow(y,2)+pow(z,2)-pow(c,2),2)+pow(pow(x,2)-1,2))*(pow(pow(z,2)+pow(x,2)-pow(c,2),2)+pow(pow(y,2)-1,2))-d }
max_gradient 5
open
contained_by { box { &lt;-1,-1,-1&gt;, &lt;1,1,1&gt;
 } }
 texture { pigment { color rgb &lt;0.564706,0.0627451,0.0627451&gt; }  } 
 finish { ambient 0.2 } 
}}
\end{verbatim}

\begin{figure}
\centering
\pandocbounded{\includegraphics[keepaspectratio]{images/APLIKOM\#3D_Sabilla Hanifah_23030630012-078.png}}
\caption{images/APLIKOM\#3D\_Sabilla\%20Hanifah\_23030630012-078.png}
\end{figure}

\textgreater povstart(angle=25°,height=10°);

\textgreater writeln(povsurface(``pow(x,2)+pow(y,2)*pow(z,2)-1'',povlook(blue),povbox(-2,2,``\,``)));

\textgreater povend();

\begin{figure}
\centering
\pandocbounded{\includegraphics[keepaspectratio]{images/APLIKOM\#3D_Sabilla Hanifah_23030630012-079.png}}
\caption{images/APLIKOM\#3D\_Sabilla\%20Hanifah\_23030630012-079.png}
\end{figure}

\textgreater povstart(angle=70°,height=50°,zoom=4);

Membuat permukaan implisit. Perhatikan sintaks yang berbeda dalam ekspresi.

\textgreater writeln(povsurface(``pow(x,2)*y-pow(y,3)-pow(z,2)'',povlook(green))); \ldots{}\\
\textgreater{} writeAxes(); \ldots{}\\
\textgreater{} povend();

\begin{figure}
\centering
\pandocbounded{\includegraphics[keepaspectratio]{images/APLIKOM\#3D_Sabilla Hanifah_23030630012-080.png}}
\caption{images/APLIKOM\#3D\_Sabilla\%20Hanifah\_23030630012-080.png}
\end{figure}

\chapter{Objek Jaring}\label{objek-jaring}

Pada contoh ini, kami menunjukkan cara membuat objek mesh, dan menggambarnya dengan informasi tambahan.

Kami ingin memaksimalkan xy di bawah kondisi x+y = 1 dan mendemonstrasikan sentuhan tangensial dari garis level.

\textgreater povstart(angle=-10°,center={[}0.5,0.5,0.5{]},zoom=7);

Kita tidak dapat menyimpan objek dalam sebuah string seperti sebelumnya, karena ukurannya terlalu besar. Jadi kita mendefinisikan objek dalam file Povray menggunakan \#declare. Fungsi povtriangle() melakukan hal ini secara otomatis. Fungsi ini dapat menerima vektor normal seperti halnya pov3d().

Berikut ini mendefinisikan objek mesh, dan menuliskannya langsung ke dalam file.

\textgreater x=0:0.02:1; y=x'; z=x*y; vx=-y; vy=-x; vz=1;

\textgreater mesh=povtriangles(x,y,z,``\,``,vx,vy,vz);

Sekarang kita tentukan dua cakram, yang akan berpotongan dengan permukaan.

\textgreater cl=povdisc({[}0.5,0.5,0{]},{[}1,1,0{]},2); \ldots{}\\
\textgreater{} ll=povdisc({[}0,0,1/4{]},{[}0,0,1{]},2);

Tuliskan permukaan dikurangi kedua cakram.

\textgreater writeln(povdifference(mesh,povunion({[}cl,ll{]}),povlook(green)));

Tuliskan kedua perpotongan tersebut.

\textgreater writeln(povintersection({[}mesh,cl{]},povlook(red))); \ldots{}\\
\textgreater{} writeln(povintersection({[}mesh,ll{]},povlook(gray)));

Tulislah satu titik secara maksimal.

\textgreater writeln(povpoint({[}1/2,1/2,1/4{]},povlook(gray),size=2*defaultpointsize));

Tambahkan sumbu dan selesaikan.

\textgreater writeAxes(0,1,0,1,0,1,d=0.015); \ldots{}\\
\textgreater{} povend();

\begin{figure}
\centering
\pandocbounded{\includegraphics[keepaspectratio]{images/APLIKOM\#3D_Sabilla Hanifah_23030630012-081.png}}
\caption{images/APLIKOM\#3D\_Sabilla\%20Hanifah\_23030630012-081.png}
\end{figure}

\chapter{Anaglyph dalam Povray}\label{anaglyph-dalam-povray}

Untuk menghasilkan anaglyph untuk kacamata merah/cyan, Povray harus dijalankan dua kali dari posisi kamera yang berbeda. Ini menghasilkan dua file Povray dan dua file PNG, yang dimuat dengan fungsi loadanaglyph().

Tentu saja, Anda membutuhkan kacamata merah/cyan untuk melihat contoh berikut dengan benar.

Fungsi pov3d() memiliki tombol sederhana untuk menghasilkan anaglyph.

\textgreater pov3d(``-exp(-x\textsuperscript{2-y}2)/2'',r=2,height=45°,\textgreater anaglyph, \ldots{}\\
\textgreater{} center={[}0,0,0.5{]},zoom=3.5);

\begin{figure}
\centering
\pandocbounded{\includegraphics[keepaspectratio]{images/APLIKOM\#3D_Sabilla Hanifah_23030630012-082.png}}
\caption{images/APLIKOM\#3D\_Sabilla\%20Hanifah\_23030630012-082.png}
\end{figure}

Jika Anda membuat scene dengan objek, Anda harus menempatkan pembuatan scene ke dalam fungsi, dan menjalankannya dua kali dengan nilai yang berbeda untuk parameter anaglyph.

\textgreater function myscene \ldots{}

\begin{verbatim}
  s=povsphere(povc,1);
  cl=povcylinder(-povz,povz,0.5);
  clx=povobject(cl,rotate=xrotate(90°));
  cly=povobject(cl,rotate=yrotate(90°));
  c=povbox([-1,-1,0],1);
  un=povunion([cl,clx,cly,c]);
  obj=povdifference(s,un,povlook(red));
  writeln(obj);
  writeAxes();
endfunction
\end{verbatim}

Fungsi povanaglyph() melakukan semua ini. Parameter-parameternya seperti pada povstart() dan povend() yang digabungkan.

\textgreater povanaglyph(``myscene'',zoom=4.5);

\begin{figure}
\centering
\pandocbounded{\includegraphics[keepaspectratio]{images/APLIKOM\#3D_Sabilla Hanifah_23030630012-083.png}}
\caption{images/APLIKOM\#3D\_Sabilla\%20Hanifah\_23030630012-083.png}
\end{figure}

\chapter{Mendefinisikan Objek sendiri}\label{mendefinisikan-objek-sendiri}

Antarmuka povray Euler berisi banyak objek. Namun Anda tidak dibatasi pada objek-objek tersebut. Anda dapat membuat objek sendiri, yang menggabungkan objek-objek lain, atau objek yang benar-benar baru.

Kami mendemonstrasikan sebuah torus. Perintah Povray untuk ini adalah ``torus''. Jadi kita mengembalikan sebuah string dengan perintah ini dan parameternya. Perhatikan bahwa torus selalu berpusat pada titik asal.

\textgreater function povdonat (r1,r2,look=``\,``) \ldots{}

\begin{verbatim}
  return "torus {"+r1+","+r2+look+"}";
endfunction
\end{verbatim}

Inilah torus pertama kami.

\textgreater t1=povdonat(0.8,0.2)

\begin{verbatim}
torus {0.8,0.2}
\end{verbatim}

Mari kita gunakan objek ini untuk membuat torus kedua, diterjemahkan dan diputar.

\textgreater t2=povobject(t1,rotate=xrotate(90°),translate={[}0.8,0,0{]})

\begin{verbatim}
object { torus {0.8,0.2}
 rotate 90 *x 
 translate &lt;0.8,0,0&gt;
 }
\end{verbatim}

Sekarang, kita tempatkan semua benda ini ke dalam suatu pemandangan. Untuk tampilannya, kami menggunakan Phong Shading.

\textgreater povstart(center={[}0.4,0,0{]},angle=0°,zoom=3.8,aspect=1.5); \ldots{}\\
\textgreater{} writeln(povobject(t1,povlook(green,phong=1))); \ldots{}\\
\textgreater{} writeln(povobject(t2,povlook(green,phong=1))); \ldots{}\\
\textgreater{}\\
\textgreater povend();

memanggil program Povray. Namun, jika terjadi kesalahan, program ini tidak menampilkan kesalahan. Oleh karena itu, Anda harus menggunakan

\textgreater povend(\textless exit);

jika ada yang tidak berhasil. Ini akan membiarkan jendela Povray terbuka.

\textgreater povend(h=320,w=480);

\begin{figure}
\centering
\pandocbounded{\includegraphics[keepaspectratio]{images/APLIKOM\#3D_Sabilla Hanifah_23030630012-084.png}}
\caption{images/APLIKOM\#3D\_Sabilla\%20Hanifah\_23030630012-084.png}
\end{figure}

Berikut adalah contoh yang lebih rumit. Kami menyelesaikan

\[Ax \le b, \quad x \ge 0, \quad c.x \to \text{Max.}\]dan menunjukkan titik-titik yang layak dan optimal dalam plot 3D.

\textgreater A={[}10,8,4;5,6,8;6,3,2;9,5,6{]};

\textgreater b={[}10,10,10,10{]}';

\textgreater c={[}1,1,1{]};

Pertama, mari kita periksa, apakah contoh ini memiliki solusi atau tidak.

\textgreater x=simplex(A,b,c,\textgreater max,\textgreater check)'

\begin{verbatim}
[0,  1,  0.5]
\end{verbatim}

Ya, benar.

Selanjutnya kita mendefinisikan dua objek. Yang pertama adalah pesawat

\[a \cdot x \le b\]\textgreater function oneplane (a,b,look=``\,``) \ldots{}

\begin{verbatim}
  return povplane(a,b,look)
endfunction
\end{verbatim}

Then we define the intersection of all half spaces and a cube.

\textgreater function adm (A, b, r, look=``\,``) \ldots{}

\begin{verbatim}
  ol=[];
  loop 1 to rows(A); ol=ol|oneplane(A[#],b[#]); end;
  ol=ol|povbox([0,0,0],[r,r,r]);
  return povintersection(ol,look);
endfunction
\end{verbatim}

Sekarang, kita bisa merencanakan adegan tersebut.

\textgreater povstart(angle=120°,center={[}0.5,0.5,0.5{]},zoom=3.5); \ldots{}\\
\textgreater{} writeln(adm(A,b,2,povlook(green,0.4))); \ldots{}\\
\textgreater{} writeAxes(0,1.3,0,1.6,0,1.5); \ldots{}\\
\textgreater{}\\
Berikut ini adalah lingkaran di sekeliling optimal.

\textgreater writeln(povintersection({[}povsphere(x,0.5),povplane(c,c.x'){]}, \ldots{}\\
\textgreater{} povlook(red,0.9)));

Dan kesalahan pada arah yang optimal.

\textgreater writeln(povarrow(x,c*0.5,povlook(red)));

Kami menambahkan teks ke layar. Teks hanyalah sebuah objek 3D. Kita perlu menempatkan dan memutarnya sesuai dengan pandangan kita.

\textgreater writeln(povtext(``Linear Problem'',{[}0,0.2,1.3{]},size=0.05,rotate=5°)); \ldots{}\\
\textgreater{} povend();

\begin{figure}
\centering
\pandocbounded{\includegraphics[keepaspectratio]{images/APLIKOM\#3D_Sabilla Hanifah_23030630012-087.png}}
\caption{images/APLIKOM\#3D\_Sabilla\%20Hanifah\_23030630012-087.png}
\end{figure}

\chapter{Contoh Lainnya}\label{contoh-lainnya}

Anda dapat menemukan beberapa contoh lain untuk Povray di Euler dalam file-file berikut.

Examples/Dandelin Spheres

Examples/Donat Math

Examples/Trefoil Knot

Examples/Optimization by Affine Scaling

\chapter{LATIHAN}\label{latihan}

\begin{enumerate}
\def\labelenumi{\arabic{enumi}.}
\tightlist
\item
  Buatlah plot 3D dari fungsi latex: f(x,y)=x\textsuperscript{3+3y}2
\end{enumerate}

dengan zoom 3 dan angle 55 derajat menggunakan povray

\textgreater pov3d(``x\textsuperscript{3+3*y}2'',zoom=3,angle=55)

\begin{figure}
\centering
\pandocbounded{\includegraphics[keepaspectratio]{images/APLIKOM\#3D_Sabilla Hanifah_23030630012-088.png}}
\caption{images/APLIKOM\#3D\_Sabilla\%20Hanifah\_23030630012-088.png}
\end{figure}

\textgreater load povray;

\textgreater defaultpovray=``C:\textbackslash Program Files\textbackslash POV-Ray\textbackslash v3.7\textbackslash bin\textbackslash pvengine.exe''

\begin{verbatim}
C:\Program Files\POV-Ray\v3.7\bin\pvengine.exe
\end{verbatim}

\begin{enumerate}
\def\labelenumi{\arabic{enumi}.}
\setcounter{enumi}{1}
\tightlist
\item
  Buatlah gabungan 2 silinder dengan fungsi povx() bewarna merah dan povz() berwarna kuning dan zoom 4
\end{enumerate}

\textgreater povstart(zoom=4)

\textgreater c1 = povcylinder (-povx,povx,1,povlook(red));

\textgreater c2 = povcylinder (-povz,povz,1,povlook(yellow));

\textgreater writeln(povintersection({[}c1,c2{]}));

\textgreater povend();

\begin{figure}
\centering
\pandocbounded{\includegraphics[keepaspectratio]{images/APLIKOM\#3D_Sabilla Hanifah_23030630012-089.png}}
\caption{images/APLIKOM\#3D\_Sabilla\%20Hanifah\_23030630012-089.png}
\end{figure}

\begin{enumerate}
\def\labelenumi{\arabic{enumi}.}
\setcounter{enumi}{2}
\tightlist
\item
  Buatlah grafik 3D dari fungsi kuadrat berikut ini dengan parameter tambahan :
\end{enumerate}

Tampilan grafik tersebut dengan transparent, dan menggunakan grid dengan resolusi 100, dengan warna kuning pada garis di plot tersebut

\textgreater plot3d(``2*x\textsuperscript{2+y}2'' ,\textgreater transparent,grid=100,wirecolor=yellow):

\begin{figure}
\centering
\pandocbounded{\includegraphics[keepaspectratio]{images/APLIKOM\#3D_Sabilla Hanifah_23030630012-090.png}}
\caption{images/APLIKOM\#3D\_Sabilla\%20Hanifah\_23030630012-090.png}
\end{figure}

4.Buatlah grafik 3D dari fungsi kuadrat berikut ini dengan parameter tambahan :

Tampilan grafik tersebut dengan transparent, dan menggunakan grid dengan resolusi 50, dengan warna biru pada garis di plot tersebut

\textgreater{} plot3d(``4*x\textsuperscript{2+2*y}2'' ,\textgreater transparent,grid=50,wirecolor=blue):

\begin{figure}
\centering
\pandocbounded{\includegraphics[keepaspectratio]{images/APLIKOM\#3D_Sabilla Hanifah_23030630012-091.png}}
\caption{images/APLIKOM\#3D\_Sabilla\%20Hanifah\_23030630012-091.png}
\end{figure}

\backmatter
\end{document}
